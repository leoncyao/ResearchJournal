\documentclass{article}
\usepackage[crop=off]{auto-pst-pdf}
\usepackage{pst-node,rotating}
\renewcommand{\familydefault}{\sfdefault}
\begin{document}
\centering 
\psset{yunit=-1}\begin{pspicture}(-0.5,-0.5)(4.0,4.25)
\psset{linewidth=2.5pt}
\rput[c](1.75,0){\textbf{rational\_experiment\_1\_small\_BNr7}}
\rput[c](1.75,0.75){}

%%%%%%%%%%%%%%%%%%%
% [[['cap', 0]], [], [0, 1]]
\psbezier(0,1.75)(0,1.25)(1,1.25)(1,1.75)
\rput[c](3.0,1.25){\color{gray}cap0}
\psbezier(0,0.75)(0,1.25)(2,1.25)(2,1.75)

%%%%%%%%%%%%%%%%%%%
% [[['neg', 0]], [0, 1], [0, 1]]
\psbezier(0,1.75)(0,2.25)(1,2.25)(1,2.75)
\psbezier[linecolor=white,linewidth=10pt](1,1.75)(1,2.25)(0,2.25)(0,2.75)
\psbezier(1,1.75)(1,2.25)(0,2.25)(0,2.75)
\rput[c](3.0,2.25){\color{gray}neg0}
\psbezier(2,1.75)(2,2.25)(2,2.25)(2,2.75)
\psline[linecolor=lightgray](3.75,1.75)(-0.25,1.75)

%%%%%%%%%%%%%%%%%%%
% [[['neg', 0]], [0, 1], [0, 1]]
\psbezier(0,2.75)(0,3.25)(1,3.25)(1,3.75)
\psbezier[linecolor=white,linewidth=10pt](1,2.75)(1,3.25)(0,3.25)(0,3.75)
\psbezier(1,2.75)(1,3.25)(0,3.25)(0,3.75)
\rput[c](3.0,3.25){\color{gray}neg0}
\psbezier(2,2.75)(2,3.25)(2,3.25)(2,3.75)
\psline[linecolor=lightgray](3.75,2.75)(-0.25,2.75)
\end{pspicture}
\end{document}

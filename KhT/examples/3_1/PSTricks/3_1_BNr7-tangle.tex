\documentclass{article}
\usepackage[crop=off]{auto-pst-pdf}
\usepackage{pst-node,rotating}
\renewcommand{\familydefault}{\sfdefault}
\begin{document}
\centering 
\psset{yunit=-1}\begin{pspicture}(-0.5,-0.5)(6.0,9.25)
\psset{linewidth=2.5pt}
\rput[c](2.75,0){\textbf{3\_1\_BNr7}}
\rput[c](2.75,0.75){}

%%%%%%%%%%%%%%%%%%%
% [[['cap', 1]], [], [1, 2]]
\psbezier(1,1.75)(1,1.25)(2,1.25)(2,1.75)
\rput[c](5.0,1.25){\color{gray}cap1}
\psbezier(0,0.75)(0,1.25)(0,1.25)(0,1.75)

%%%%%%%%%%%%%%%%%%%
% [[['neg', 0]], [0, 1], [0, 1]]
\psbezier(0,1.75)(0,2.25)(1,2.25)(1,2.75)
\psbezier[linecolor=white,linewidth=10pt](1,1.75)(1,2.25)(0,2.25)(0,2.75)
\psbezier(1,1.75)(1,2.25)(0,2.25)(0,2.75)
\rput[c](5.0,2.25){\color{gray}neg0}
\psbezier(2,1.75)(2,2.25)(2,2.25)(2,2.75)
\psline[linecolor=lightgray](5.75,1.75)(-0.25,1.75)

%%%%%%%%%%%%%%%%%%%
% [[['neg', 0]], [0, 1], [0, 1]]
\psbezier(0,2.75)(0,3.25)(1,3.25)(1,3.75)
\psbezier[linecolor=white,linewidth=10pt](1,2.75)(1,3.25)(0,3.25)(0,3.75)
\psbezier(1,2.75)(1,3.25)(0,3.25)(0,3.75)
\rput[c](5.0,3.25){\color{gray}neg0}
\psbezier(2,2.75)(2,3.25)(2,3.25)(2,3.75)
\psline[linecolor=lightgray](5.75,2.75)(-0.25,2.75)

%%%%%%%%%%%%%%%%%%%
% [[['neg', 0]], [0, 1], [0, 1]]
\psbezier(0,3.75)(0,4.25)(1,4.25)(1,4.75)
\psbezier[linecolor=white,linewidth=10pt](1,3.75)(1,4.25)(0,4.25)(0,4.75)
\psbezier(1,3.75)(1,4.25)(0,4.25)(0,4.75)
\rput[c](5.0,4.25){\color{gray}neg0}
\psbezier(2,3.75)(2,4.25)(2,4.25)(2,4.75)
\psline[linecolor=lightgray](5.75,3.75)(-0.25,3.75)

%%%%%%%%%%%%%%%%%%%
% [[['cap', 2]], [], [2, 3]]
\psbezier(2,5.75)(2,5.25)(3,5.25)(3,5.75)
\rput[c](5.0,5.25){\color{gray}cap2}
\psbezier(0,4.75)(0,5.25)(0,5.25)(0,5.75)
\psbezier(1,4.75)(1,5.25)(1,5.25)(1,5.75)
\psbezier(2,4.75)(2,5.25)(4,5.25)(4,5.75)
\psline[linecolor=lightgray](5.75,4.75)(-0.25,4.75)

%%%%%%%%%%%%%%%%%%%
% [[['pos', 1]], [1, 2], [1, 2]]
\psbezier(2,5.75)(2,6.25)(1,6.25)(1,6.75)
\psbezier[linecolor=white,linewidth=10pt](1,5.75)(1,6.25)(2,6.25)(2,6.75)
\psbezier(1,5.75)(1,6.25)(2,6.25)(2,6.75)
\rput[c](5.0,6.25){\color{gray}pos1}
\psbezier(0,5.75)(0,6.25)(0,6.25)(0,6.75)
\psbezier(3,5.75)(3,6.25)(3,6.25)(3,6.75)
\psbezier(4,5.75)(4,6.25)(4,6.25)(4,6.75)
\psline[linecolor=lightgray](5.75,5.75)(-0.25,5.75)

%%%%%%%%%%%%%%%%%%%
% [[['neg', 2]], [2, 3], [2, 3]]
\psbezier(2,6.75)(2,7.25)(3,7.25)(3,7.75)
\psbezier[linecolor=white,linewidth=10pt](3,6.75)(3,7.25)(2,7.25)(2,7.75)
\psbezier(3,6.75)(3,7.25)(2,7.25)(2,7.75)
\rput[c](5.0,7.25){\color{gray}neg2}
\psbezier(0,6.75)(0,7.25)(0,7.25)(0,7.75)
\psbezier(1,6.75)(1,7.25)(1,7.25)(1,7.75)
\psbezier(4,6.75)(4,7.25)(4,7.25)(4,7.75)
\psline[linecolor=lightgray](5.75,6.75)(-0.25,6.75)

%%%%%%%%%%%%%%%%%%%
% [[['cup', 3]], [3, 4], []]
\psbezier(3,7.75)(3,8.25)(4,8.25)(4,7.75)
\rput[c](5.0,8.25){\color{gray}cup3}
\psbezier(0,7.75)(0,8.25)(0,8.25)(0,8.75)
\psbezier(1,7.75)(1,8.25)(1,8.25)(1,8.75)
\psbezier(2,7.75)(2,8.25)(2,8.25)(2,8.75)
\psline[linecolor=lightgray](5.75,7.75)(-0.25,7.75)
\end{pspicture}
\end{document}
